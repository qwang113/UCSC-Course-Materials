\documentclass[11pt, twocolumn]{asaproc}

\usepackage{graphicx}
\usepackage{times}
\usepackage{lipsum}

\title{Title of the data analysis report}

\author{Your name}
\begin{document}


\maketitle

\begin{abstract}
A short paragraph summarizing the aim of the report, what you did, and what you found.
\end{abstract}



\section{Primary Subhead\label{intro}}

\lipsum[1]

\begin{table*}[htb]
\caption{A 2-column table.}
\begin{center}
\begin{tabular}{ccccc}
\hline
\hline
\\[-5pt]
\multicolumn{2}{c}{Genotype} & &
\multicolumn{1}{c}{Dummy for additivity} &
\multicolumn{1}{c}{Dummy for dominance }\\
\multicolumn{1}{c}{Label} &    
\multicolumn{1}{c}{Index i} &
\multicolumn{1}{c}{Genotypic value ($\eta$)}&
\multicolumn{1}{c}{effect $\alpha$ (x)} &
\multicolumn{1}{c}{effect $\delta$ (z)}\\
\hline
qq      &1&     $\mu + \mbox{2}\alpha$  & 2&    0\\
Qq&     2&      $\mu + \alpha + \delta$&        1       &1\\
QQ&     3&      $\mu$&  0&      0\\
\hline
\end{tabular}
\end{center}
\end{table*}


\subsection{Secondary Subhead}


\lipsum[1]

\subsection{Secondary Subhead}


\lipsum[1]

\section{Another Primary Subhead}

\subsection{Secondary Subhead}

\begin{figure*}[htb]
\centering\includegraphics[width = 0.8\textwidth]{matrix.png}
\caption{A 2-column figure.}
\end{figure*}


\lipsum[1]

\subsubsection{Tertiary Subhead}


\lipsum[1]



\lipsum[1]

\begin{figure}[htb]
\centering\includegraphics[width = \columnwidth]{matrix.png}
\caption{A 1-column figure}
\end{figure}

\lipsum[1]

\end{document}


