\documentclass[11pt]{article}
\usepackage{fullpage,amssymb}
\pagestyle{empty}
\usepackage{color}
\usepackage[colorlinks=true,urlcolor=blue]{hyperref}
\newcommand{\alert}[1]{\textcolor{red}{#1}}


\begin{document}
\begin{center}
{
\Large {\bf STAT 7L \hfill Computer Lab for Statistics (STAT 7) \hfill Summer 2022}
}

\vspace{.3in}
{
\Large {\bf Course Policies and Syllabus}
}
\vspace{.25in}

\begin{tabular}{rcc} 
{\bf Instructor:} Qi Wang \\
{\bf Email:} qwang113@ucsc.edu \\
% {\bf Zoom Office Hours (via Canvas):} dfsdfdsfdsfds\\
\end{tabular}
\end{center}

\vspace{.2in}

\noindent% TODO
% {\bf Lab Sessions:} \\ \\
% \begin{tabular}{l l l l l}
%      \alert{Days}   &  \alert{Times}      & \alert{Location} & \alert{Instructor} & *** \\  
%      \alert{Days}   &  \alert{Times}      & \alert{Location} & \alert{Instructor} & *** \\  
%      \alert{Days}   &  \alert{Times}      & \alert{Location} & \alert{Instructor} & *** \\  
%      \alert{Days}   &  \alert{Times}      & \alert{Location} & \alert{Instructor} & *** \\  
%      \alert{Days}   &  \alert{Times}      & \alert{Location} & \alert{Instructor} &  \\  
%      \alert{Days}   &  \alert{Times}      & \alert{Location} & \alert{Instructor} &  \\  
%      \alert{Days}   &  \alert{Times}      & \alert{Location} & \alert{Instructor} &  \\  
%      \alert{Days}   &  \alert{Times}      & \alert{Location} & \alert{Instructor} &  \\  
% \end{tabular}
% \vspace{0.15in}

% *** These labs will only be offered the first mandatory occurrence,
% \alert{date} and \alert{date}. After that, you will need to attend another lab for drop-in
% help. If no other lab fits your schedule, contact \alert{Instructor 1} or \alert{Instructor 2} to schedule
% an appointment.
% \vspace{0.15in}

% \noindent \underline{Please attend the first lab for orientation.} Subsequent lab attendance is NOT
% required.
% \vspace{0.3in}

\noindent {\bf Web page:} All announcements and lab assignments are in {\em Canvas}. Login to your {\em Canvas} using your GoldID and password and enter STAT-7L. 
The login page for {\em Canvas} can be accessed using the following URL: 
% \begin{center}
\url{http://canvas.ucsc.edu}.
% \end{center} 
\vspace{0.3in}

\noindent {\bf Associated lectures:} \\
STAT 7, Mendes,B.S., TuTh 06:00PM-09:30PM. (Oakes Acad 105)

\vspace{.3in}

\noindent {\bf Lab sessions}\\ 
Remote (Recorded): Will be posted every Monday and Wednesday morning.\\

\noindent {\bf Office Hours }

Monday: 8:00 - 10:00 a.m.  and Wednesday: 8:00 - 10:00 a.m. \\


\vspace{.3in}

%\noindent {\bf Text: }{\em Biostatistics for the Biological
%  and Health Sciences},
%M. M. Triola and M. F. Triola, Pearson (2006), or recommended text for STAT7

\noindent {\bf Tentative textbook: }

We are using the same book as STATS 7 classes.

\vspace{.3in}

\noindent {\bf Course objectives:} To acquire the technological skills
needed to implement methods learned in STAT 7 using the statistical
software \href{https://www.jmp.com/en_us/home.html}{JMP}, and to reinforce various concepts from STAT 7 through
computer simulation and data analysis.  \vspace{.2in}

\noindent {\bf Lab assignments (100\%):} Lab assignments will be completed,
submitted, and reviewed in {\em Canvas}.  All lab assigments will be posted in the
{\em Modules} section. Assignments will be comprised of {\em Canvas Quizzes} and {\em iMathAS} 
question sets. Data files used in labs can be accessed in the {\em Modules} themselves, 
but they are also available for download in the {\em Files} tab on {\em Canvas}.

Labs will be posted and due once a week. Labs will be  due the following
Monday at 11:59PM after being posted previous Monday. 

Labs are self-paced and do NOT have a time limit; however, ALL labs
MUST be submitted by the posted due dates. Most lab assignments will consist of
multiple sections, each of which you will be expected to complete, submit,
and review one at a time before starting the next section of the lab. 
You do not have to complete all lab
parts in one session. {\em Canvas Quizzes} allow you to save parts and return
to complete them at a later time, but {\em iMathAS} question sets do not. It is 
recommended that you record all answers by hand if you wish to return to an 
assignment at a later time. Labs are designed to take approximately 90 minutes to complete all parts combined,
but may be shorter or longer depending on your familiarity with the material.
Each part of a lab may be submitted twice, with the highest score
counting toward your grade. You are allowed and encouraged to work on labs
alongside your peers, but every student is expected to do their own
calculations and JMP analysis required by the lab. Submitting work not
completed by you is a violation of academic integrity. For more information on academic dishonesty, 
refer to the Academic Dishonesty section below.
\vspace{0.2in}

\noindent {\bf Late work:} Late submissions will {\bf NOT} be accepted. The
class accommodates missing Lab assignments by designating Labs 5 and 10 as
extra credit (see {\bf Course Grade} section below). Therefore, instructors
will adhere to a strict assignment submission policy. Complete the labs early
in the week. Do not wait until the day the assignments are due! In cases of
extenuating circumstances (typically requiring documentation, i.e. doctor's note), accommodating late work will be left at the
discretion of the instructors. In such cases, email both instructors at least
48 hours before the due date of the assignment.
\vspace{0.2in}
 
\noindent {\bf Student support:} Students are encouraged to use the Canvas discussion boards to ask questions and 
collaborate with other students. The instructor will monitor these and respond to questions regularly. 
For more personal matters, students may also email the instructor at any time throughout the course. 
Emails may be sent directly to an instructor or by using the {\em Messages} tool in {\em Canvas}. 
Note that last minute emails may not be answered
immediately. Be sure to send your inquiries to instructors well before the
due date (don't wait until the night before to do the lab). 
% Zoom office hours will also be held each week according to the schedule above.  
Individual Zoom meetings or in-person appoitments may be possible on a case-by-case basis.
\vspace{0.2in}

\noindent {\bf Tentative schedule and content list:}  \vspace{.05in}  \\ 
\indent \begin{tabular}{| l | l |p{10cm}|} \hline
Lab \# & Due Date & Content \\ \hline
Lab 1 & 08/01/2022, 11:59 pm &Practice with Data Types, Starting JMP. \\ \hline

Lab 2 & 08/18/2022, 11:59 pm & Looking at data. Measures of central tendency, Measures of dispersion. \\ \hline

Lab 3 & 08/15/2022, 11:59 pm &Relative Frequency, Probability (including Bayes Theorem), Binomial and Poisson distribution.\\ \hline

Lab 4 & 08/22/2022, 11:59 pm & Means of Normals, Central Limit Theorem, Normal Approximation to Binomial.\\ \hline

Lab 5.1 & 08/29/2022, 11:59 pm & EXTRA CREDIT. \\ \hline

% Lab 9 & 05/31/2021, 11:59 pm & Multiple Regression, Goodness-of-Fit Tests\\ \hline

Lab 5.2 & 08/29/2021, 11:59 pm & EXTRA CREDIT. \\ \hline
% Two-Sample Tests for Means, Polynomial Regression, Optimization.\\ \hline
\end{tabular}

\vspace{.3in}

\noindent {\bf Course grade:} Grades will be based on a point system. Each
question within a required lab is worth one point. The primary
grade percentage will be calculated out of the total. Extra credit labs carry
an additional 20\% total, more than enough to replace an entire missed lab.
The final score (the raw percentage plus extra credit) will determine a
student's letter grade: 90\% - 100\% is an A, 80\% - 89\% is a B, 70\% - 79\%
is a C, 60\% - 69\% is a D, and 0 - 59\% is an F. Note that A+ will not be
given for students who finish higher than 100\%. We will not round or bargain
for scores that are borderline between different grade levels. \\

\noindent {\bf DRC accommodations:} The Disability Resources Center reduces barriers to inclusion and full participation for students
with disabilities by providing support to individually determine reasonable academic
accommodations. If you have questions or concerns about accommodations or any other
disability-related matter, please contact the DRC office, located in Hahn 125 or at 831-459-2089
or \href{mailto:drc@ucsc.edu}{drc@ucsc.edu}. \\

\noindent {\bf Academic dishonesty:} Academic integrity is the cornerstone of a university education. Academic dishonesty diminishes
the university as an institution and all members of the university community. It tarnishes the
value of a UCSC degree.

All members of the UCSC community have an explicit responsibility to foster an environment of
trust, honesty, fairness, respect, and responsibility. All members of the university community are
expected to present as their original work only that which is truly their own. All members of the
community are expected to report observed instances of cheating, plagiarism, and other forms
of academic dishonesty in order to ensure that the integrity of scholarship is valued and
preserved at UCSC.

In the event a student is found in violation of the UCSC Academic Integrity policy, he or she may
face both academic sanctions imposed by the instructor of record and disciplinary sanctions
imposed either by the provost of his or her college or the Academic Tribunal convened to hear
the case. Violations of the Academic Integrity policy can result in dismissal from the university
and a permanent notation on a students transcript.

For the full policy and disciplinary procedures on academic dishonesty, students and instructors
should refer to the \href{https://www.ue.ucsc.edu/academic_misconduct}{Academic Integrity page} at the Division of Undergraduate Education. \\

\noindent {\bf Title IX:} The university cherishes the free and open exchange of ideas and enlargement of knowledge.
To maintain this freedom and openness requires objectivity, mutual trust, and confidence; it
requires the absence of coercion, intimidation, or exploitation. The principal responsibility for
maintaining these conditions must rest upon those members of the university community who
exercise most authority and leadership: faculty, managers, and supervisors.
The university has therefore instituted a number of measures designed to protect its community
from sex discrimination, sexual harassment, sexual violence, and other related prohibited
conduct. \href{https://titleix.ucsc.edu/}{Information about the Title IX Office}, the \href{https://ucsc-gme-advocate.symplicity.com/public_report/index.php/pid681212?}{online reporting link}, applicable campus
\href{https://titleix.ucsc.edu/Resources\%20and\%20Options\%202.18.pdf}{resources}, reporting responsibilities, the \href{https://policy.ucop.edu/doc/4000385/SVSH}{UC Policy on Sexual Violence and Sexual Harassment}
and the UC Santa Cruz Procedures for Reporting and Responding to Reports of Sexual
Violence and Sexual Harassment can be found at \href{titleix.ucsc.edu}{titleix.ucsc.edu}.
The Title IX/Sexual Harassment Office is located at 105 Kerr Hall. In addition to the \href{https://ucsc-gme-advocate.symplicity.com/public_report/index.php/pid681212?}{online
reporting option}, you can contact the Title IX Office by calling 831-459-2462. 

\end{document}
